\chapter{Results}
The seventh chapter, \textit{Results}, presents the outcome of this Bachelor thesis. The first section, \textit{Summary}, summarises the results of the chapters \textit{Analysis}, \textit{Design}, \textit{Implementation} and \textit{Evaluation}. The second section, \textit{Limitations}, names and explains the limitations of this Bachelor thesis. 

\section{Summary}
To better understand the problem of finding a parked car in an unknown environment, four users are interviewed about their experiences with the problem. The interviewees report that they most often have to look for their car in big parking lots and in multi-storey parking lots. When asked about a potential application to help users to find their parked car, the interviewees name most often two features. First, they expect such an application to be accurate. The majority agrees that an accuracy of better than 50 meters would be sufficient. Second, they expect the application to automatically determine the parking position of the car, without any manual user input. The need for manual user input would prevent some of the users from using the application. Based on the results of the interviews, five functional and seven non-functional requirements are defined for an application to support users to find their parked car.

A prototype for the user interface design is developed based on the defined requirements. It is improved iteratively in four iterations using the user-centric Think-Aloud method. Between each iteration, the current prototype is showed to users which are asked to perform five different tasks. The key results of the interviews are aggregated and summarised. Based on the results of the interviews, the next prototype is improved. 

The final user interface consists of three main views: ''Map View - Loading'', ''Map View - Parking Position Determined'', and ''Feedback''. The first view, ''Map View - Loading'', is shown when the application is started. It shows a map with the current position of the user clearly marked and a button to initiate the determination of the parking position of the user's car. The second view, ''Map View - Parking Position Determined'', is shown after the application successfully determined the parking position of the user's car. It presents the determined position on a map together with the current position of the user. The view also shows detailed information about the determined position, such as the address, the relative altitude and distance from the user's current location to the determined parking position. The user can initiate from this view a third party navigation with directions to the determined car position and they can initiate the reporting of the accuracy of the determined parking position. The view ''Map View - Loading'' supports the user in reporting the determined parking position.

To determine the parking position of the user's car based on the analysis of their spatial trajectory, an algorithm is developed. This algorithm uses the user's spatial trajectory. The trajectory is cleared from noise. Based on the stay points of the trajectory, it is cut into individual trips. These trips are individually evaluated for parking position candidates of the user's car. To determine a potential parking position in a trip, the transportation mode for each moving window segment set is evaluated using a machine learning model. The location where the user last stopped traveling by car is a potential parking position. The most recent of these potential parking positions is used as the determined parking position of the user's car. 

The developed algorithm to determine the parking position of a user's car depends on a machine learning model to classify the transportation mode of a user's trajectory. To create this model, data from the GeoLife project by Microsoft Research Asia is used. The data is prepared and cleaned. The features velocity, acceleration and bearing change are created for each individual segment of each trajectory. Aggregated features, such as the minimum, the maximum, the range, the sum, the average, and the variance, are created based on the individual features for each moving window set of segments. Based on the created data set, two machine learning models are trained: a Multilayered Perceptron classifier, which is a class of neural networks, and a XGBoost classifier, which is a greedy tree-boosting algorithm. 

Based on the design described above, an application is implemented. The application is developed using the programming language Swift and runs exclusively on Apple iPhones. The application logs the spatial trajectory of the user and stores it locally on the iPhone using the database Realm. The designed algorithm to determine the parking position of the user's car is implemented in the application and is not dependent on any server structure. No private data, such as any location data, is sent to any server element to ensure the privacy of the user. To be able to evaluate the accuracy of the determined parking position, the application enables the user to report the accuracy of the determined parking position to a server. All sent information are shown to the user before they are sent and they do not contain any sensitive data, such as the location data or any information that could be used to identify the user. 

The outcome of this Bachelor thesis is evaluated in three ways. First, the performances of the trained machine learning models are compared. The Multilayer Perceptron classifier reaches an accuracy of 89.8\%. The XGBoost classifier reaches an accuracy of 90.48\%. The parameters for both classifiers are optimised using three-fold cross validation. The XGBoost classifier is used in the implementation of the application as it shows the better results. Second, the accuracy of the determined parking position is evaluated. Thirteen tests are performed by four people. The mean accuracy is 112.7 meters with a standard deviation of 85.12 meters. Third, the fulfillment of the defined requirement is evaluated. The implemented application fulfills all five functional requirements and 6 of the eight non-functional requirements. The missing two non-functional requirements are fulfilled partly. 


\section{Limitations}
% Analysis
The analysis of the problem is conducted with semi-structured interviews. For these interviews four people are interviewed. For a more representative survey significantly more people need to be interviewed. 

% Design
To design an user-friendly user interface Think-Aloud interviews are conducted between each of the four iterations. At each iteration only one person is interviewed. To ensure a good usability for a wide range of users, more people need to be interviewed. 

The Think-Aloud interviews are performed in an office environment. To improve the results of these interviews even more they can be performed in the context in which the application is used. 

For the classification of the transportation mode of the user's spatial trajectory, two machine learning models are trained. To find the best machine learning mode for this task, more models can be trained.

The trained classifiers can only distinguish between the transportation modes walking and car. In real world scenarios users might use other transportation modes, such as bikes, subways and buses, which are not covered by the developed system. 

% Implementation
The developed application is implemented for Apple iPhones and cannot easily be adapted for Android devices. The designed approach of the application to determine the parking position of the user's car can be used to implement an Android application. 

% Evaluation
For the evaluation of the accuracy of the determined parking position, 13 data points are applied. To get a more accurate evaluation of the accuracy, more tests have to be conducted. 


