\chapter{Detailed Interview Results}
Appendix A, \textit{Detailed Interview Results}, presents the detailed results of the interviews conducted for analysing the problem of finding a parked car and the expectations of the users. The answers of the interviewees to each question are summarized in short statements and presented in tables. The asked questions are divided into three sections.
In the first section of questions, \textit{Problem Domain}, the interviewees are asked about their experiences with finding the parking position of their car.
In the second section of questions, \textit{Application}, the interviewees are asked about their expectations of an application, which helps them to find their parked car.
In the third section of questions, \textit{Background}, the interviewees are asked questions about their background.

\section{Problem Domain}

\newcommand{\yes}{\textbf{\cellcolor{lightgray!60}yes}}
\newcommand{\no}{no}
\begin{table}
    \begin{tabular}{|c|c c c c|} 
        \toprule
        Answer & I1 & I2 & I3 & I4 \\
        \midrule
        Big parking lot & \no & \yes & \yes & \yes\\
        Multi-storey parking lot & \yes & \no & \no &\yes \\
        Unknown area & \no & \no & \no & \yes\\
        Big street, after not driving the car for some days & \yes & \no & \no & \no\\
        \bottomrule 
    \end{tabular}
    \caption{In which situations in your past could you not find your parked car?}
    \label{table:a:1}
\end{table}

\begin{table}
    \begin{tabular}{|c|c c c c|} 
        \toprule
        Answer & I1 & I2 & I3 & I4 \\
        \midrule
        Walk around in rough area of car & \yes & \yes & \yes & \yes \\
        Use remote key for audio feedback & \no & \no & \yes & \yes\\
        Use saved information, such as parking level & \yes & \no & \no & \no \\
        Use orientation points, such as unique buildings & \no & \yes & \no & \no \\
        \toprule
    \end{tabular}
    \caption{Imagine you cannot remember where exactly you parked your car. Could you describe how you would approach finding your car?}
    \label{table:a:1}
\end{table}


\section{Application}

\section{Background}



\begin{table}
    \begin{tabular}{|c|c c c c|} 
        \toprule
        Answer & I1 & I2 & I3 & I4 \\
        \midrule
        
        \toprule
    \end{tabular}
    \caption{Question}
    \label{table:a:1}
\end{table}