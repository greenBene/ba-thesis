\chapter{Analysis}
In the third chapter, \textit{Analysis}, the needs and expectations of users for an application to determine the parking position of a car are identified and requirements for such an application are defined.
In the first section, \textit{Background Interviews}, potential users are interviewed to identify the needs and expectations they have for an application to determine the parking position of their car.
In the second section, \textit{Requirements}, the requirements of the application are defined based on the results of the background interviews and the constraints of this Bachelor thesis.

\section{Background Interviews}

To identify the needs and expectations of users, four users are interviewed for background interviews. The background interviews are semi-structured. All people who drive cars are potential users of the application. Because the application solves a specific problem that is applicable to most drivers, the target audience is not segmented into smaller groups. Four people are interviewed who are part of the target audience. The interviewed people are called participants. All participants volunteer for the interview without any extrinsic incentive, and agree to be recorded and quoted anonymously. \cite{Abras2004} \cite{wilson2013interview}

To identify the questions a nomenclature is introduced. The questions are identified with ''[Q.xx]'', where Q stands for question, ''.'' separates the letter from the number and x is a number starting at 01. The brackets ''[]'' limit the term.

\paragraph{[Q.01] In which situations in your past could you not find your parked car?}
The two most mentioned locations where users have trouble finding their car are big parking lots, mentioned by three of the four participants, and multi-storey parking lots, mentioned by two of the four participants.

\paragraph{[Q.02] Imagine you cannot remember where exactly you parked your car. Could you describe how you would approach finding your car?}
When asked to describe their approach to find their parked car if the exact location is not known, three out of four participants say they would walk around the area where the car should be. Two out of four would use the remote key of the car to unlock and lock the car to create visual and auditory feedback to indicate if the car is nearby.

\paragraph{[Q.03] Please list the two or three features which are most important for you in an application that helps you to find your parked car.}
The two most important aspects of the application, each mentioned by two of the four participants, are the accuracy of the determined parking position and the independence from manual user input, such as the manual logging of the parking position.

\paragraph{[Q.04] How accurate do you expect the application to be?}
All of the participants express that an accuracy of 10 meters is sufficient for the determined parking position. 
Two of the four participants are also satisfied with an accuracy of 50 meters.

\paragraph{[Q.05] Which information should the application present to you?}
All participants expect the application to show the position of their parked car. Two out of four participants also expect to see their current position and the position of their parked car on a map. 

\paragraph{[Q.06] What would prevent you from using the application?}
Required manual user input or a complicated user interface would each prevent two of four participants to use the application. A lack of clarity about the use of the generated user data and bad reviews in the AppStore would each prevent one participant to use the application.

\paragraph{[Q.07] How do you feel about the application using your location data?}
For three of the four participants the use of the location data is not an issue. The remaining participant accepts the usage of their location data if the application openly communicates how the generated data is used.

\paragraph{[Q.08] Would it be reassuring for you to know that no location data is sent to the cloud? }
All participant would be reassured if the location data is only used on the device itself, but none of them define this aspect as a requirement. They would also use the application, if the data is uploaded and processed online. 

\paragraph{[Q.09] Since when do you drive cars?}
Two of the participants drive cars since eighth years. The other two participants drive since six and twelve years.

\paragraph{[Q.10] How regularly do you drive cars?}
Three of four participants drive currently only a few times a year by car. One participant drives weekly. Two of four mention that they used to drive daily. The other two participants mention that they used to drive monthly or weekly. 

\paragraph{[Q.11] Did you use applications which help you to find your car before?}
Only one of the four participants previously used an application to help them find their car. This person used Google Maps.

\paragraph{[Q.12] Are you aware that Google Maps and Apple Maps offer a feature to help finding a parked car?}
Only one of the participants is aware that Google Maps and Apple Maps provide the functionality to assist in finding a user's parked car. 

\paragraph{[Q.13] Do you drive more often in cities or rural areas? A city is defined as a town with a population greater than 100.000 people.}
Two of the participants report that they drive more often in urban areas. The other two either drive in rural and urban areas equally or primarily in rural areas.


\section{Requirements}
The requirements are split into functional and non-functional requirements. A functional requirement ''specifies a function that a system or system component must be able to perform.''. A non-functional requirement is defined as a requirement that is not a functional requirement, such as data requirements, constraints and quality requirements. \cite{eide2005quantification}

To identify the requirements, a nomenclature is introduced. The requirements are uniquely identified with ''[XX.yy]''. XX can either be equal to ''RF'' or ''RN''. ''RF'' stands for functional requirement and ''RN'' stands for non-functional requirement. ''yy'' is a positive natural number starting at 01. ''.'' separates the letters from the number. The brackets ''[]'' limit the term.  

\subsection{Functional Requirements}

\paragraph{[RF.01] The application determines the parking position of the user's car using spatial trajectory analysis}
The parking position is determined by analysing the spatial trajectory of the user. The spatial trajectory, based on the GPS sensor of the phone, is segmented and classified into the used transportation mode of the user and a heuristic is used to determine the parking position based on this segmentation. The requirement is built upon the constraints of this Bachelor thesis. 

\paragraph{[RF.02] The application presents the determined parking position of the user's car on a map}
The parking position of the user's car is presented on a map to enable the user to easily orientate themselves in the environment. The requirement is based on the users' answers to [Q.05].

\paragraph{[RF.03] The application presents the current position of the user on a map}
The current user location is presented on a map to enable the user to orient themselves in their current environment. The requirement is based on the users' answers to [Q.05].

\paragraph{[RF.04] The application presents detailed information about the determined parking position, namely the address, the distance from the current location and the relative altitude}
The detailed information of the determined parking location is presented to the user to enable them to find the most fitting way to reach their parked car. The requirement is based on the users' answers to [Q.05].

\paragraph{[RF.05] The application enables the user to report the accuracy of the determined parking position to the developer}
To enable the user to report the accuracy of the determined parking position, a view is implemented. The accuracy is defined as the distance between the determined parking position of the user's car and the actual parking position. This requirement is necessary to enable the quantitative evaluation of the parking position accuracy. 

\subsection{Non-functional Requirements}

\paragraph{[RN.01] The application is easy to use}
The user interface of the application does not confuse the user and presents all options in an easy to understand way. The user is always aware of the state of the application. This requirement is based on the users' answers to [Q.06].

\paragraph{[RN.02] The application does not send any location data to server components of the implemented system}
To ensure the privacy of the user and to avoid collecting sensible information, the application does not send any location data to any server component of the implemented system. The requirement is based on the users' answers to [Q.08].

\paragraph{[RN.03] The application does not rely on manual user input for determining the parking position}
The application can determine the parking position of the user's car without the user being asked to manually input any data, such as the parking position or the transportation mode chosen. The process to determine the parking position is fully automated. The requirement is necessary due to half of the users reporting in [Q.06] that manual user input would prevent them from using the application.

\paragraph{[RN.04] The accuracy of the determined parking position is better than 50 meters}
The accuracy, defined as the distance between the determined parking position of the user's car and its actual parking position, is better than 50 meters. Users report the accuracy as one of the most important features of the application in [Q.03] and the majority agrees that an accuracy of better than 50 meters is sufficient in [Q.04]. 

\paragraph{[RN.05] The application is developed to run on iOS 13}
The application is developed to run on iPhones with the latest iPhone operating system iOS 13. The requirement is based on the constraints of this Bachelor thesis. 

\paragraph{[RN.06] The application is functional on big parking lots}
One of the most named locations in [Q.01] where users had problems finding their parked car are big parking lots, such as in front of major supermarkets or furniture stores. Thus, the application needs to support the user in finding their parked car in the mentioned environments.

\paragraph{[RN.07] The application is functional on multi-storey parking lots}
The other most named location in [Q.01] where users had problems finding their parked car are multi-storey parking lots. Thus, the application needs to support the user in finding their parked car in these environments by providing data to determine the floor on which the car is parked. 