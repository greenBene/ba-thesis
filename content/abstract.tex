\begin{abstractDE}

In Großstädten sind Parkplätze schwer zu finden, da die Anzahl an privaten Fahrzeugen zunimmt und die verfügbare Parkfläche nicht effektiv genutzt ist. Fahrer haben Probleme ihr geparktes Auto wiederzufinden und nutzen dadurch Parkplätze länger als nötig während sie ihr Auto suchen. Ein System, welches Fahrer dabei unterstützt ihr Auto zu finden, würde dabei helfen unnötig genutzte Parkplätze schneller frei zu geben und darum die Parkplatznot in Großstädten lindern. In der vorliegenden Bachelorarbeit wird eine mobile Applikation entworfen und implementiert, welche die Parkposition des Autos eines Fahrers automatisch anhand seiner Raumtrajektorie bestimmen kann. Um die Position des Autos zu bestimmen, wird ein Algorithmus entworfen, der auf die Klassifikation der Transportationsmethode der Raumtrajektorie des Nutzers aufbaut. Die Applikation ist mit nutzerzentrierten Methoden entworfen.

\end{abstractDE}

\vfill

\begin{abstractEN}

Parking spaces in major cities are hard to find as the number of private vehicles increases and the available land for parking spaces is not efficiently used. As drivers have sometimes difficulties to find their parked car and thus use a parking spot longer than necessary while searching for their car, a system to support the user to find their parked car can help to reduce the lack of available parking spaces. In this Bachelor thesis an application to determine the parking position of a user's car is designed and implemented. To determine the position of the user's parked car, an algorithm is designed that is based on the classification of the transportation mode of the user's spatial trajectory. The application is designed using user-centric methods. 

\end{abstractEN}

\vfill