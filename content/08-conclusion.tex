\chapter{Conclusion}
The eigth chapter, \textit{Conclusion}, discusses the results of this Bachelor thesis and potential future work. In the first section, \textit{Discussion of Results}, the results of this Bachelor thesis are discussed. In the second section, \textit{Future Work}, possible future works related to this Bachelor thesis are presented. 

\section{Discussion of Results}


The background interviews conducted in this Bachelor Thesis show that users experience the problem of finding a parked car but the majority of the users is not aware that there are existing solutions for this problem implemented in the most used navigation applications, such as Google Maps and Apple Maps. This shows that the developers of major navigation applications are aware that the problem exists and already started to develop systems to solve it, but may not be confident in their solutions as they do not present them prominently in their applications.

The developed algorithm to determine the parking position of a user's car based on the spatial trajectory of the user does successfully determine the parking position in 46\% of the tests. This indicates that the spatial trajectory analysis of the user's trajectory is a viable way to determine the parking position of a user's car. Still, none of the major navigation applications uses this approach. This could be caused by the relative high battery impact of the constant trajectory recording. 

The transportation mode classification in this Bachelor thesis is similar to the approaches used in the most influential related works. A major difference is the omission of any post-processing. The developed system still shows a similar accuracy compared to the related work. This shows that the classification of the transportation mode does not need any post-processing. 

The developed application successfully implements the designed algorithm to determine the parking position of a user's car based on the spatial trajectory of the user. It performs the classification of the user's trajectory in near real time. This shows how performant modern smartphones are and underlines that more computation can be executed on the user's device itself. 


\section{Future Work}
The developed approach focuses on the transportation modes car and walk. Future work can use a similar approach for other transportation modes, such as bike, more suitable for other environments. The developed approach is adaptable for such changes. 

This Bachelor thesis constrains itself to the use of the user's spatial trajectory to determine the parking position of their car. Future work can use more information, such as the acceleration data of the user's smartphone, to determine the parking position. 

In future work the developed algorithm to determine the parking position of the user's car can be implemented for other smartphones devices, such as Android devices. This allows the evaluation of the algorithm on a more diverse set of hardware. 

