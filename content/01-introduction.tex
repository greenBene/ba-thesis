\chapter{Introduction}

The first chapter, \textit{Introduction}, introduces the domain of parking spaces, explains why it is useful to find a parked car in a short period of time and summarises the contribution of this Bachelor thesis for finding a parked car based on trajectory analysis.
The first section, \textit{Motivation}, explains the problem of overused parking spaces in major cities. It shows why it can be difficult to find a parked car and how the period of time a driver needs to find their parked car affects the number of unoccupied parking spaces.
The second section, \textit{Outline of Contribution}, summarises the contribution of the Bachelor thesis and the functionality of the developed application, which supports the drivers in finding their parked car. 
The third section, \textit{Structure of Thesis}, describes the structure of the Bachelor thesis.

\section{Motivation}
% introduce the domain
The first section, \textit{Motivation}, explains the problem of overused parking spaces in major cities. It shows why it can be difficult to find a parked car and how the period of time a driver needs to find their parked car affects the number of unoccupied parking spaces.

Finding an unoccupied parking space in a major city is becoming more difficult as the number of private vehicles increases and the available land for parking spaces is not efficiently used. It is estimated that 30\% of driving vehicles in cities are searching for a parking space. A decrease in parking time would lead to more available parking spaces, as the parking spaces would be less in unnecessary use and thus would be earlier available again. \cite{wu2007robust} \cite{Ibrahim2018} \cite{Geng2012}

% introduce the problem 
Drivers are more conscious about the appointment they are heading to than the exact location where they park. The frustration caused by the difficulty to find an unoccupied parking space increases this effect. Hence, it can be time-consuming to find the position where the car is parked after the appointment. Parking spaces are longer occupied than necessary.
  
% give an easy example
\textit{\textbf{Example 1}
Alexandra Smith has been invited for a job interview at Boho Technologies in London. The public transport in the village she lives in is not optimal, so she decides to go to the interview by car. Half an hour before the appointment she is at the company and is looking for a parking space. After 10 minutes she finds one in a small side street. She parks there and makes it just in time to the job interview. After the successful appointment, she cannot remember where her car is parked. She needs more than 30 minutes to find it. When she finds it, she notices an 80£ fine on her car because she stayed longer than allowed on the parking space.
}

As \textit{Example 1} shows that it is challenging to find a parked car in a major city. This leads to frustrated drivers and the parking spaces are longer occupied than necessary. A system to support the drivers at finding their cars is needed to help reducing the unnecessary utilisation of highly needed parking spaces in major cities.

\section{Outline of Contribution}
The second section, \textit{Outline of Contribution}, summarises the contribution of the Bachelor thesis and the functionality of the developed application, which supports the drivers in finding their parked car. 

% Artefact
In the Bachelor thesis an iPhone\footnote{iPhone: \url{https://www.apple.com/iphone/}, (last accessed October 15, 2019)} application is developed, which supports drivers of cars to locate their parked car. The application uses spatial trajectories, based on the Global Positioning System (GPS), to model each trip of the driver to determine the parking position of the car. The application shows the location of the parked car and the location of the user on a map. All calculations involving the spatial trajectories of the users are done on the mobile device to ensure privacy.

% Method
To enable the application to determine the parking position of a car, an algorithm is developed to determine the parking position of the users car based on the users spatial trajectory. The spatial trajectory of the user is split into trips based on the stay points of the trajectory. Based on the transportation mode of the user on a trip segment, the parking position of their car is determined. The transportation mode is classified using a machine learning model. 

\section{Structure of Thesis}
The third section, \textit{Structure of Thesis}, describes the structure of the Bachelor thesis by introducing each chapter with its function in the thesis.

Chapter two, \textit{Background}, defines terms necessary to understand the Bachelor thesis, explains the used methodology for analysing the problem and discusses related systems and related academic works.
In the third chapter, \textit{Analysis}, background interviews are conducted to identify the needs and expectations of potential users for an application that determines the parking position of a car. Based on the results of the interviews and the constraints of the Bachelor thesis, requirements are defined for the application.
The fourth chapter, \textit{Design}, presents the results of the user interface design, the created algorithm to determine the parking position of a users car and the used machine learning workflow.
The fifth chapter, \textit{Implementation}, describes the implementation of the training of the machine learning model used for classifying the transportation of the user and the implementation of the developed application. 
The sixth chapter, \textit{Evaluation}, evaluates the results of the Bachelor thesis.
The seventh chapter, \textit{Results}, presents the outcome of the Bachelor thesis.
The eight chapter, \textit{Conclusion}, discusses the results of the Bachelor thesis and potential future work based on the Bachelor thesis.